\begin{prob}
	در محیط
	\texttt{prob}
	می‌توان صورت مسئله یا پاسخ آن را نوشت.  در انتهای سؤال یک خط افقی رسم می‌شود و در ابتدای آن یک خط‌چین. به طور پیش‌فرض سؤال‌ها از سؤال ۱ شروع شده و یکی یکی اضافه می‌شوند.
	
\end{prob}
\begin{sol}
	محیط
	\texttt{sol},
	در انتهای آخرین خط متنی که درون آن قرار دارد یک 
	$ \blacksquare $
	قرار می‌دهد.\\	
	همچنین پس از اتمام متن، یک خط افقی رسم می‌کند.
\end{sol}
\begin{probnum}{10}
	می‌توان با استفاده از محیط
	\texttt{probnum}
	سؤال با شمارهٔ دل‌خواه نوشت و پس از آن با استفاده از محیط 
	\texttt{prob}
	شمارش سؤالات به طور معمول خواهد بود.
	
\end{probnum}

\begin{prob}
	شماره‌گذاری سؤال‌ها به صورت عادی ادامه پیدا می‌کند.
	
\end{prob}
\begin{customEnv}{محیط دل‌خواه}
	می‌توان از محیط
	\texttt{customEnv}
	برای مواردی مثل «لم»، «ادعا» و... استفاده کرد. برای شماره‌گذاری دل‌خواه هم می‌توان از
	\texttt{customEnvnum}
	استفاده کرد. متن درون این محیط به صورت ایتالیک می‌باشد. در انتهای این محیط یک نقطه‌چین قرار می‌گیرد.
\end{customEnv}
\begin{proof}
	پس از محیطی مانند
	\textbf{لم}
	یا
	\textbf{ادعا}
	معمولاً لازم است اثباتی برای آن آورده‌شود.\\
	مشابه محیط راه‌حل، در انتهای محیط اثبات یک 
	$\square$
	در پایان آخرین خط قرار می‌گیرد. پس از متن اثبات نیز نقطه‌چین اضافه می‌شود.
\end{proof}